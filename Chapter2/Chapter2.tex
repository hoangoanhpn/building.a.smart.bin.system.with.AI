% !TEX root = ..\thesis.tex

\chapter{CƠ SỞ LÍ THUYẾT}
\label{chap:ontology}
\section{Giới thiệu TensorFlow và TensorFlow Lite}
%TensorFlow Lite is an open source deep learning framework for on-device inference.
TensorFlow Lite là giải pháp gọn nhẹ của TensorFlow cho thiết bị di động và thiết bị nhúng.
Nó cho phép suy luận học máy trên thiết bị với độ trễ thấp và kích thước nhị phân nhỏ.

\section{ESP AI Thinker}

\section{Bộ dữ liệu huấn luyện và thử nghiệm}


Giới thiệu dataset TrashNet \cite{trashnet}


\section{Lý thuyết mạng CNN}
Mạng CNN là một tập hợp các lớp Convolution chồng lên nhau và sử dụng các hàm nonlinear activation như ReLU và tanh để kích hoạt các trọng số trong các node. CNN được dùng trong trong nhiều bài toán như nhân dạng ảnh, phân tích video, ảnh MRI, hoặc cho bài các bài của lĩnh vự xử lý ngôn ngữ tự nhiên. Trong đề tài này, chúng tôi đã sử dụng mạng CNN để giải quyết bài toán phân loại rác.
Bài toán có đầu vào là một hình ảnh của rác được đưa vào thùng và đầu ra là nhãn của loại rác đó.
CNN bao gồm tập hợp các lớp cơ bản bao gồm: convolution layer + nonlinear layer, pooling layer, fully connected layer. 
Các lớp này liên kết với nhau theo một thứ tự nhất định. 
Thông thường, một ảnh sẽ được lan truyền qua tầng convolution layer + nonlinear layer đầu tiên, sau đó các giá trị tính toán được sẽ lan truyền qua pooling layer, bộ ba convolution layer + nonlinear layer + pooling layer có thể được lặp lại nhiều lần trong network. Và sau đó được lan truyền qua tầng fully connected layer và softmax để tính xác suất ảnh đó chứa vật thế gì.

\section{Giới thiệu công nghệ LoRaWan}
LoRa(long-range) sử dụng kỹ thuật điều chế gọi là Chirp Spread Spectrum.
Có thể hiểu nôm na nguyên lý này là dữ liệu sẽ được băm bằng các xung cao tần để tạo ra tín hiệu có dãy tần số cao hơn tần số của dữ liệu gốc (cái này gọi là chipped); sau đó tín hiệu cao tần này tiếp tục được mã hoá theo các chuỗi chirp signal (là các tín hiệu hình sin có tần số thay đổi theo thời gian; 
Có 2 loại chirp signal là up-chirp có tần số tăng theo thời gian và down-chirp có tần số giảm theo thời gian; và việc mã hoá theo nguyên tắc bit 1 sẽ sử dụng up-chirp, và bit 0 sẽ sử dụng down-chirp) trước khi truyền ra anten để gửi đi.

Theo Semtech công bố thì nguyên lý này giúp giảm độ phức tạp và độ chính xác cần thiết của mạch nhận để có thể giải mã và điều chế lại dữ liệu; hơn nữa LoRa không cần công suất phát lớn mà vẫn có thể truyền xa vì tín hiệu Lora có thể được nhận ở khoảng cách xa ngay cả độ mạnh tín hiệu thấp hơn cả nhiễu môi trường xung quanh.
Băng tần làm việc của LoRa từ 430MHz đến 915MHz cho từng khu vực khác nhau trên thế giới:

430MHz cho châu Á

780MHz cho Trung Quốc

433MHz hoặc 866MHz cho châu Âu

915MHz cho USA

LoRaWAN là giao thức mạng năng lượng thấp, diện rộng (LPWA) được phát triển bởi Liên minh LoRa, kết nối không dây ‘hoạt động’ với internet trong các mạng khu vực, quốc gia hoặc toàn cầu, nhắm mục tiêu các yêu cầu chính của Internet of Things (IoT) như bi thông tin liên lạc hai chiều, dịch vụ bảo mật đầu cuối, di động và nội địa hóa.
LoRaWAN sử dụng phổ không được cấp phép trong các dải ISM để xác định giao thức truyền thông và kiến ​​trúc hệ thống cho mạng trong khi lớp vật lý LoRa tạo ra các liên kết giao tiếp tầm xa giữa các cảm biến từ xa và các cổng kết nối với mạng. Giao thức này giúp thiết lập nhanh chóng các mạng IoT công cộng hoặc riêng tư ở bất cứ đâu bằng phần cứng và phần mềm.

