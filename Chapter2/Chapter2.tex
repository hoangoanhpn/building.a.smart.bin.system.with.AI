% !TEX root = ..\thesis.tex

\chapter{CƠ SỞ LÍ THUYẾT}

% các cách sử dụng để giải quyết bài toán
\section{Giới thiệu bài toán phân loại ảnh}
Phân loại ảnh là một bài toán phổ biến trong lĩnh vực thị giác maý tính. Mục tiêu của bài toán là làm cho máy tính có khả năng xác định nhãn của hình ảnh. Cụ thể hơn bài toán có đầu vào và đầu ra như sau:

- Đầu vào: Ảnh chứa đối tượng cần xác định nhãn và danh sách các nhãn (labels).

- Đầu ra: nhãn tương ứng với đối tượng trong ảnh đầu vào.

% kiếm đại hình phân loại ảnh nào cũng được, có liên quan đến đề tài càng tốt
Hình ... minh họa đầu vào và đầu ra của bài toán. Bài toán này được áp dụng trong nhiều lĩnh vực như: phân loại biển báo giao thông, phân loại chữ viết tay,... Các phương pháp giải quyết bài toán được chia làm hai loại: tiếp cận dựa trên máy học và học sâu. Đối với cách tiếp cận dựa trên máy học trước tiên cần phải xác định các đặc trưng từ hình ảnh bằng một số phương pháp như: Scale-invariant feature transform (SIFT), Histogram of oriented gradients (HOG). Sau đó sử dụng thêm một số kỹ thuật phân lớp như thuật toán Support vector machine (SVM) hoặc Decision Tree để phân loại đối tượng. Cách tiếp cận dựa trên học sâu sử dụng kiến trúc mạng Convolution neural network cho cả việc trích xuất đặc trưng và phân loại đối tượng. Ngoài ra ta có thể kết hợp cả hai phương pháp bằng cách sử dụng mạng Convolution neural networkd để trích xuất đặc trưng sau đó sử dụng các kỹ thuật phân lớp để phân loại đối tượng. 

Trong những năm gần đây, sự phát triển của khoa học kỹ thuật và lượng dữ liệu ngày càng lớn đã tạo điều kiện cho các kiến trúc mạng CNN được áp dụng ngày càng nhiều trong việc giải quyết các bài toán phân loại ảnh. Các mô hình này cũng mang lại kết quả cao với thời gian ngắn đủ để đáp ứng yêu cầu áp dụng trong các bài toán real-time của con người. Chính vì vậy chúng tôi đã sử dụng thuật toán CNN cho khóa luận này.

\section{Giới thiệu CNN}s
CNN là một kiến trúc mạng bao gồm các lớp: convolution layer + nonlinear layer, pooling layer, fully connected layer liên kết với nhau theo một thứ tự nhất định. Thông thường, ảnh được truyền qua lớp convolution + nonlinear sau đó đến pooling layer. Bộ ba convolution + nonlinear và pooling layer được lặp lại nhiều lần trong mạng. Sau đó các giá trị tính toán được lan truyền qua tầng fully connected và softmax để phân loại đối tượng trong ảnh. Hình ... minh họa mạng CNN cơ bản.

Trong kiến trúc mạng này ảnh đầu vào được biểu diễn dưới dạng ma trận. Lớp Convolution được sử dụng để phát hiện đặc trưng cụ thể của ảnh, từ các đặc trưng cơ bản như góc, cạnh đến các đặc trưng phức tạp như texture của ảnh. Một ma trận có kích thước nhỏ (3x3 hoặc 5x5) được gọi là filter sẽ lướt qua toàn bộ ảnh từ trái sang phải và từ trên xuống dưới để phát hiện các đặc trưng trong ảnh. Hình ... minh họa cụ thể phép tính convolution. 

Kích thước của filter tỉ lệ thuận với số tham số cần học và thường là số lẻ. Sau khi áp dụng phép convolution thì ma trận đầu vào sẽ nhỏ dần dẫn đến số layer của mô hình CNN bị giới hạn. Vì vậy cần phải lưu ý đến tham số stride, thể hiện số pixel cần phải dịch chuyển mỗi khi trượt filter trên ảnh và thêm padding có giá trị bằng 0 ở phần khung của ảnh. Tương tự như mạng neurald network, CNN cũng sử dụng hàm kích hoạt như ReLU hoặc tanh đặt ngay sau tầng convolution. Đối với dữ liệu ảnh hàm kích hoạt thường được sử dụng là ReLU, hàm này gán những giá trị âm bằng 0 và giữ nguyên các giá trị lớn hơn 0.

Pooling layer được xếp sau các lớp convolution để giảm tham số. Các loại pooling được sử dụng chủ yếu là max pooling và average. Các lớp này sẽ được lặp lại theo thứ tự để tạo ra feature map cuối cùng sau đó truyền vào tầng fully connected. Tầng này chuyển ma trận nhận được thành vector và phân loại vector đó tương tự như mạng neural network.

% dùng CNN giải quyết bài toán gì, lý thuyết về CNN, còn các phần khác của QUyết đưa vào mục dưới ( mục đóng góp)
% Mạng CNN là một tập hợp các lớp Convolution chồng lên nhau và sử dụng các hàm nonlinear activation như ReLU và tanh để kích hoạt các trọng số trong các node. CNN được dùng trong trong nhiều bài toán như nhân dạng ảnh, phân tích video, ảnh MRI, hoặc cho bài các bài của lĩnh vự xử lý ngôn ngữ tự nhiên. Trong đề tài này, chúng tôi đã sử dụng mạng CNN để giải quyết bài toán phân loại rác.
% Bài toán có đầu vào là một hình ảnh của rác được đưa vào thùng và đầu ra là nhãn của loại rác đó.
% CNN bao gồm tập hợp các lớp cơ bản bao gồm: convolution layer + nonlinear layer, pooling layer, fully connected layer. 
% Các lớp này liên kết với nhau theo một thứ tự nhất định. 
% Thông thường, một ảnh sẽ được lan truyền qua tầng convolution layer + nonlinear layer đầu tiên, sau đó các giá trị tính toán được sẽ lan truyền qua pooling layer, bộ ba convolution layer + nonlinear layer + pooling layer có thể được lặp lại nhiều lần trong network. Và sau đó được lan truyền qua tầng fully connected layer và softmax để tính xác suất ảnh đó chứa vật thế gì.

\label{chap:ontology}
\section{Giới thiệu TensorFlow và TensorFlow Lite}
%TensorFlow Lite is an open source deep learning framework for on-device inference.
TensorFlow Lite là giải pháp gọn nhẹ của TensorFlow cho thiết bị di động và thiết bị nhúng.
Nó cho phép suy luận học máy trên thiết bị với độ trễ thấp và kích thước nhị phân nhỏ.



\section{Bộ dữ liệu huấn luyện và thử nghiệm}


Giới thiệu dataset TrashNet \cite{trashnet}


\section{Giới thiệu công nghệ LoRaWan}
LoRa(long-range) sử dụng kỹ thuật điều chế gọi là Chirp Spread Spectrum.
Có thể hiểu nôm na nguyên lý này là dữ liệu sẽ được băm bằng các xung cao tần để tạo ra tín hiệu có dãy tần số cao hơn tần số của dữ liệu gốc (cái này gọi là chipped); sau đó tín hiệu cao tần này tiếp tục được mã hoá theo các chuỗi chirp signal (là các tín hiệu hình sin có tần số thay đổi theo thời gian; 
Có 2 loại chirp signal là up-chirp có tần số tăng theo thời gian và down-chirp có tần số giảm theo thời gian; và việc mã hoá theo nguyên tắc bit 1 sẽ sử dụng up-chirp, và bit 0 sẽ sử dụng down-chirp) trước khi truyền ra anten để gửi đi.

Theo Semtech công bố thì nguyên lý này giúp giảm độ phức tạp và độ chính xác cần thiết của mạch nhận để có thể giải mã và điều chế lại dữ liệu; hơn nữa LoRa không cần công suất phát lớn mà vẫn có thể truyền xa vì tín hiệu Lora có thể được nhận ở khoảng cách xa ngay cả độ mạnh tín hiệu thấp hơn cả nhiễu môi trường xung quanh.
Băng tần làm việc của LoRa từ 430MHz đến 915MHz cho từng khu vực khác nhau trên thế giới:

430MHz cho châu Á

780MHz cho Trung Quốc

433MHz hoặc 866MHz cho châu Âu

915MHz cho USA

LoRaWAN là giao thức mạng năng lượng thấp, diện rộng (LPWA) được phát triển bởi Liên minh LoRa, kết nối không dây ‘hoạt động’ với internet trong các mạng khu vực, quốc gia hoặc toàn cầu, nhắm mục tiêu các yêu cầu chính của Internet of Things (IoT) như bi thông tin liên lạc hai chiều, dịch vụ bảo mật đầu cuối, di động và nội địa hóa.
LoRaWAN sử dụng phổ không được cấp phép trong các dải ISM để xác định giao thức truyền thông và kiến ​​trúc hệ thống cho mạng trong khi lớp vật lý LoRa tạo ra các liên kết giao tiếp tầm xa giữa các cảm biến từ xa và các cổng kết nối với mạng. Giao thức này giúp thiết lập nhanh chóng các mạng IoT công cộng hoặc riêng tư ở bất cứ đâu bằng phần cứng và phần mềm.

